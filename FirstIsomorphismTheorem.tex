\documentclass{article}

\usepackage{amsmath}
\usepackage{amsfonts}

\renewcommand{\Im}{\text{Im}}

\begin{document}
	\section*{First Isomorphism Theorem}
	Let $f:G\rightarrow H$ be a group homomorphism. The
	\begin{itemize}
		\item $K=\ker(f):={g\in G|f(g)=1_H}$ is a normal subgroup of $G$,
		\item $\Im(f)$ is a subgroup of $H$, and
		\item $G/K$ is isomorphic to $\Im(f)$.
	\end{itemize}
	In fact, the isomorphism is given by$$
		\bar{f}:G/K\rightarrow\Im(f),\ \ \bar{f}(\bar{g})=\bar{f}(gK)=f(g)
	$$
	\subsection*{$\bar{f}$ is well defined}
	Let $\bar{g}_1,\bar{g}_2\in G/K$, the statement is true iff$$
		\forall \bar{g}_1,\bar{g}_2\in G/K,\ \ \bar{g}_1=\bar{g}_2 \rightarrow \bar{f}(\bar{g}_1)=\bar{f}(\bar{g}_2)
	$$
	The proof is given by\begin{align*}
		&\bar{g}_1=\bar{g}_2 \implies g_1K=g_2K \implies g_1^{-1}g_2\in K\\
		\implies& f(g_1^{-1}g_2)=1_H \implies f(g_1^{-1})f(g_2)=1_H\\
		\implies& f(g_1)=f(g_2)
	\end{align*}
	\subsection*{$K$ is a subgroup of $G$}
	\subsubsection*{Identity}
	Since $f$ is a homomorphism, $f(1_G)=1_H$, therefore $1_G\in\ker(f)$
	\subsubsection*{Inverse}
	To prove: $\forall k\in K,\ k^{-1}\in K
	$\\
	Let $k\in K$, $f(kk^{-1})=f(1_G)=1_H=f(k)f(k^{-1})=1_Hf(k^{-1})$\\
	As $1_H=1_Hf(k^{-1})$,$f(k^{-1})=1_H$
	\subsubsection*{Operation}
	To prove: $
		\forall k_1,k_2\in K,\ k_1k_2\in K
	$\\
	Given any such $k_1,k_2$, since $f$ is an isomorphism, we have\begin{align*}
		&f(k_1k_2)\\
		=&f(k_1)f(k_2)\\
		=&1_H1_H\\
		=&1_H
	\end{align*}
	Therefore $k_1k_2\in K$.
	\subsection*{$K$ is normal}
	By definition, $K$ is a normal subgroup of $G$ iff $$
		\forall k\in K, g\in G,\ \ gkg^{-1}\in K
	$$
	Given any $k,g$, since $f$ is an isomorphism, we have\begin{align*}
		&f(gkg^{-1})\\
		=&f(g)f(k)f(g^{-1})\\
		=&f(k)
	\end{align*}
	Therefore $gkg^{-1}\in K$.
	\subsection*{$\Im(f)$ is a subgroup of $H$}
	\subsubsection*{Identity}
	Since $G$ is a group, it has identity. Since $f$ is a homomorphism, $f(1_G)=1_H$. Therefore $1_H\in\Im(f)$
	\subsubsection*{Inverse}
	To prove: $
		\forall h\in\Im(f), h^{-1}\in\Im(f)
	$\\
	Let $h\in\Im(f)$, then $\exists g\in G$ s.t. $f(g)=h$ and because $f$ is a homomorphism, $f(g^{-1})=h^{-1}$
	\subsubsection*{Operation}
	To prove: $
		\forall h_1,h_2\in\Im(f), \ h_1h_2\in\Im(f)
	$\\
	Let $h_1 = f(g_1)$ and $h_2 = f(g_2)$, since $f$ is an isomorphism, \begin{align*}
		h_1h_2 = f(g_1)f(g_2) = f(g_1g_2)
	\end{align*}
	As shown above, there $\exists$ something $\in G$ that maps to $h_1h_2$ by $f$.
	\subsection*{$\bar{f}$ is an homomorphism}
	What we need to prove to prove the statement is
	$$
		\forall \bar{g}_1,\bar{g}_2\in G/K,\ \ 
		\bar{f}(\bar{g}_1\bar{g}_2)=\bar{f}(\bar{g}_1)\bar{f}(\bar{g}_2)
	$$
	The proof is given by\begin{align*}
		&\bar{f}(\bar{g}_1\bar{g}_2)\\
		=&\bar{f}((g_1K)(g_2K))\\
		=&\bar{f}((g_1g_2K))\\
		=&f(g_1g_2)\\
		=&f(g_1)f(g)\\
		=&\bar{f}(\bar{g}_1)\bar{f}(\bar{g}_2)
	\end{align*}
	\subsection*{$\bar{f}$ is surjective}
	$\Im(\bar{f})=\{\bar{f}(\bar{g})|\bar{g}\in G/K\}=\{\bar{f}(gK)|g\in G\}=\{f(g)|g\in G\}=\Im(f)$\\
	If $h\in\Im(f)$ then $h\in\Im(\bar{f})$.
	\subsection*{$\bar{f}$ is injective}
	Let $\bar{g}_1,\bar{g}_2\in G/K$, the proof is given by \begin{align*}
		&\bar{f}(\bar{g}_1)=\bar{f}(\bar{g}_2) \\
		\implies &
		\bar{f}(\bar{g}_1^{-1})\bar{f}(\bar{g}_1)=\bar{f}(\bar{g}_1^{-1})\bar{f}(\bar{g}_2) \\
		\implies &
		\bar{f}(\bar{g}_1^{-1})\bar{f}(\bar{g}_2)=1\\
		\implies & f(g^{-1}g)=1\\
		\implies & g^{-1}g\in K\\
		\implies & g_1K=g_2K\\
		\implies & \bar{g}_1=\bar{g}_2
	\end{align*}
\end{document}